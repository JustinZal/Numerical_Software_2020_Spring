\documentclass[a4paper,12pt]{article}

\usepackage[english]{babel}

\begin{document}
\title{Numerical Software Lab Project}
\author{Justinas Zaliaduonis}
\maketitle


\section{Introduction and Motivation}

Projects are part of the Numerical Software Lab at Jacobs University. Tasks were completed using \textbf{Python3} programming language and open-source scientific computing software packages \textbf{numpy}, \textbf{matplotlib}, and \textbf{scipy}.

\section{Project A}
\textbf{Goal:} Create a function that takes TXT file name as an input, extracts first 2 columns from the file as a \textit{numpy} array, graph them on an X-Y axis, saves the graph to a PDF file converts data to CSV file and returns the input as 2 separate arrays. \\ \\ \textbf{Execution: } The goal was achieved by creating a function, that takes in \textit{file\_name} as an input and performs the required manipulations and plotting. The function also has default inputs \textit{save, title, x\_label, y\_label}, set to True, 'title', 'x\_label', 'y\_label' respectively, which allow function execution without any files saved and enables additional graph labeling functionality. Output is returned as a tuple of arrays.

\section{Project B}

\textbf{Goal:} Create a function that 3D-plots specified Spherical Bessel function. \\ \\ \textbf{Execution:} The goal was achieved by creating a function, that takes in \textit{x\_range} and \textit{y\_range} variable in the form of list. \textit{Ex:} [2, 10], \textit{function\_type} (1 or 2) and function order \textit{n} in the form of integer and plots asked function into the file named \textit{projectB.pdf}.


\section{Project C}

\textbf{Goal:} Write a function, that solves \textit{van der Pol oscillator} with the specified differential equation and plots the solution relationships. \\ \\ \textbf{Execution:} The goal was achieved by creating a function named \textit{solve\_oscilator\_equation}, that takes in integers \textit{miu}, \textit{A} and \textit{omega}, that serve as constants for the differential equation. Function also takes required parameters \textit{start} and \textit{end} as indicators for solution neighborhood and optional parameters \textit{x0} and \textit{v0} as initial conditions for the differential equation, that default to 1 and 0 respectively. \textit{x versus time}, \textit{v versus time} and \textit{v versus x} relationships are graphed and saved into a file named \textit{projectC.pdf}. 


\section{Project D}

\textbf{Goal:} Write functions, that \textbf{1)} Generates a required matrix, \textbf{2)} Extracts eigenvalues from a matrix, \textbf{3)} Iterates function \textbf{1)} with different input values, passes input values to the function \textbf{2)} and then plots eigenvalues on a complex coordinate plane. \\ \\ \textbf{Execution:} Goal \textbf{1)} was achieved by creating a function \textit{systemmatrix} that takes a required real number \textit{d} and an optional real number \textit{k} (defaults to -1000) and returns a required matrix in the form of \textit{numpy} array. Goal \textbf{2)} was achieved by creating a function named \textit{get\_eigen\_values}, that takes in \textit{matrix} in the form of \textit{numpy} array, and then utilizes \textit{scipy} package to return eigenvalues of a matrix. Goal \textbf{3)} Was achieved by writing a function named \textit{projectD}, that takes in 2 optional natural numbers \textit{N} and \textit{M}, which default to 100 and 500 respectively. Number M indicates the range of iteration, and number N indicates the density of iteration. After the iteration is completed, all eigen values are collected, real and complex values separated and ploted on a complex number plane. The plot is saved in the file named \textit{projectD.pdf}.

\section{Project E}

\textbf{Goal:} Write a function that takes output of a function \textit{projectA} as input and plots data, fit function and cubic interpolation of the input. \\ \\ \textbf{Execution:} The goal was achieved by creating a function named \textit{projectD}, that takes in the output of a function \textit{projectA} as input in the form of \textit{x\_array} and \textit{y\_array}. The function also takes in additional optimal parameters of \textit{title}, \textit{x\_axis} and \textit{y\_axis} to specify the plot details. If optional arguments are not passed, the plot will be without title or labels. All 3 plots are plotted onto one graph and saved in a file named \textit{projectE.pdf}. The fit function was chosen of the form $f(x)=ax^{3}+c$. In the example, function \textit{projectA} is called with a parameter Save = False, otherwise it would overwrite previously created pdf and csv files. 

\section{Project F}

\textbf{Goal: } Write a function, that takes output from \textit{projectC} as input, and makes a Fourier transform of t with respect to x.

\section{Remarks}

- Before every plot function \textit{plt.clf()} is called to clean the previous plotting memory 

\end{document}